\documentclass[a4paper,oneside,openany,12pt]{memoir}
\usepackage{graphicx} % For images
%\usepackage[english]{babel} % For correct word hyphenating.
\usepackage{color} % For the colored boxes.
\usepackage{hyperref} % For URLs.
\usepackage{float} % For custom floats (the info boxes).

% Include all of this in a separate file!
\newcommand{\HRule}{\rule{\linewidth}{1mm}} % Doet height (zie style) hetzelfde?
\newcommand{\institute}[1]{\gdef\inst{#1}}  % Beamer supplies \institute. We want that, too.
\newcommand{\inst}{}                        % Beamer supplies \institute. We want that, too.
\newcommand{\funding}[1]{\gdef\fund{#1}}    % Provide funding line.
\newcommand{\fund}{}                        % Provide funding line.
\newcommand{\setLOVDversion}[1]{\gdef\LOVDversion{#1}} % Provide the current LOVD version.
\newcommand{\LOVDversion}{}                            % Provide the current LOVD version.

\setlrmarginsandblock{2cm}{2cm}{*} % LEFT-RIGHT
\setulmarginsandblock{2cm}{2cm}{*} % TOP-BOTTOM
\checkandfixthelayout % Without this, nothing works. Took me ages before I found out.
\fixpdflayout % Not sure if we need this, but it was recommended someplace.



\makepagestyle{LOVD}

% Because we don't have odd or even pages, we only need to define odd pages.
\makeoddhead{LOVD}{\normalfont\leftmark}{}{\normalfont\rightmark}
\makeheadrule{LOVD}{\textwidth}{\normalrulethickness}
\makeoddfoot{LOVD}{}{\normalfont\thepage}{}
\makefootrule{LOVD}{\textwidth}{\normalrulethickness}{\footruleskip}

% Style "plain" is called from chapters. We want chapters to have a footer as well.
\makeoddfoot{plain}{}{\normalfont\thepage}{}
\makefootrule{plain}{\textwidth}{\normalrulethickness}{\footruleskip}

% Additional changes:
\makepsmarks{LOVD}{%
  \nouppercaseheads

  \createmark{chapter}{left}{shownumber}{}{.\space} % (\leftmark) number, followed by a . and a space.
  \createmark{section}{right}{nonumber}{}{.\space} % (\rightmark) no number, (useless: followed by a . and a space).
  % Change "shownumber" to "nonumber" if you don't want the chapter/section number displayed at the header.

%  \createplainmark{toc}{both}{\contentsname}
%  \createplainmark{lof}{both}{\listfigurename}
%  \createplainmark{lot}{both}{\listtablename}
%  \createplainmark{bib}{both}{\bibname}
%  \createplainmark{index}{both}{\indexname}
%  \createplainmark{glossary}{both}{\glossaryname}
  % Might want to keep those, see the manual for further information.
}
% Activate your new pagestyle
\pagestyle{LOVD}



%\newcommand{\maketitle}{%
%  \vspace*{\droptitle}
%  \maketitlehooka
%  {\pretitle \title \posttitle}
%  \maketitlehookb
%  {\preauthor \author \postauthor}
%  \maketitlehookc
%  {\predate \date \postdate}
%  \maketitlehookd
%  \thispagestyle{title}
%}

\setlength{\droptitle}{-3cm} % Moves the title (logo, title, authors etc) 3cm up.
\pretitle{
  \begin{center}
    \includegraphics[width=17cm]{../gfx/logo.jpg}
    \vskip 4cm
    \HRule\par\HUGE\bfseries\sffamily} %% Need proper font!
\posttitle{\par\HRule\end{center}\vskip 3cm}
\preauthor{\flushright}
\postauthor{\par\inst\par\vskip 1cm}
\predate{\hfill Last updated } % \hfill aligns the rest of the line to the right. \flushright would have done the same.
\postdate{\par\vskip 2cm \noindent \small \fund}

\makechapterstyle{LOVD}{%
  \setlength\beforechapskip{10pt} % A small distance just above the new Chapter title.
  \setlength\afterchapskip{20pt} % A small distance between the Chapter title and the text.
  \renewcommand{\chapterheadstart}{\vspace*{\beforechapskip}\hrule height 2pt \medskip} % Nice ruler above the Capter title.
  \renewcommand{\chapnamefont}{\normalfont\large\scshape} % CHAPTER
  \renewcommand{\chapnumfont}{\normalfont\LARGE\scshape} % 1
  \renewcommand{\chaptitlefont}{\normalfont\huge\bfseries\scshape} % e.g. "Introduction"
  \renewcommand{\printchaptername}{} % Empty text instead of "Chapter".
                  \renewcommand{\chapternamenum}{ } % Weet niet wat dit anders doet.
                  \renewcommand{\printchapternum}{\chapnumfont \thechapter} % Weet niet wat dit anders doet.
  \renewcommand{\afterchapternum}{. } % Just a dot after the Chapter number, no new line.
  \renewcommand{\afterchaptertitle}{\par\nobreak\medskip\hrule\vskip\afterchapskip} % Nice ruler below the Capter title.
}
\chapterstyle{LOVD}

% For the colored boxes.
%\definecolor{gray}{rgb}{0.9,0.9,0.9}
%\definecolor{darkgray}{gray}{0.3}
\definecolor{linkblue}{rgb}{0.1, 0, 1}
\hypersetup{
  colorlinks,
  citecolor=linkblue,
  filecolor=linkblue,
  linkcolor=linkblue,
  urlcolor=linkblue
}

% The custom floats.
\floatstyle{ruled}
\newfloat{infobox}{h!}{floats}
\floatname{infobox}{}

\floatstyle{boxed}
\newfloat{notebox}{h!}{floats}
\floatname{notebox}{Note:}



\setLOVDversion{3.0-beta-02}
\title{LOVD 3.0 user manual \\\vskip 1cm Build \LOVDversion}
\author{Ivo F.A.C. Fokkema}
\institute{Leiden University Medical Center}
\date{2012-02-17} % I guess it's easier to use this as a "Last modified" column.
\funding{LOVD has received funding from the European Community's Seventh Framework Programme (FP7/2007-2013) under grant agreement no 200754 - the GEN2PHEN project.}





\begin{document}

\begin{titlingpage} % We don't want the front to count as page 1.
\maketitle
\end{titlingpage}





\tableofcontents





\chapter{Introduction}
This is the manual for the Leiden Open (source) Variation Database (LOVD) version 3.0.
LOVD 3.0 is a partial rewrite of LOVD 2.0, which first stable release was completed in 2007.
Also, LOVD 3.0 has a greatly improved database model, and includes lots of new features aimed at making LOVD useful for more research environments.
\par
LOVD is designed to provide a flexible, freely available tool for gene-centered collection and display of DNA variations.
LOVD 3.0 extends this idea to also provide patient-centered data storage and storage of NGS data, even of variants outside of genes.
\\\par
LOVD was developed approaching the ``LSDB-in-a-Box'' idea for the easy creation and maintenance of a fully web-based gene sequence variation database,
that is platform-independent and uses PHP and MySQL open source software only.
The design of the database follows the recommendations of the \href{http://www.hgvs.org/}{Human Genome Variation Society} (HGVS)
and focuses on the collection and display of DNA sequence variations, but it has fully implemented methods for storing complete clinical data as well.
The open LOVD setup also facilitates functional extensions with scripts written by the community.
\\\par
The development of (then nameless) LOVD started in late 2002, while it was first officially released in January, 2004.
Before that LOVD was only in use by the \href{http://www.DMD.nl/}{Leiden Muscular Dystrophy pages},
as a not-so-modular system with lots of characteristics specific for that website only.
With the official release of LOVD in 2004 the system had become much more dynamic and customizing LOVD was made easy mostly by editing text-files.
\par
In 2004, LOVD became available under the open source license GPL and with the 1.1.0 release most of the text-files had been replaced by online forms
so customisations can be performed through the web interface.
Early in 2005 the \href{http://www.ncbi.nlm.nih.gov/pubmed/15977173}{first LOVD article} was published,
and in 2005 the development of LOVD was more targeted at improving the ease of use of the system.
\\\par
In 2006 the development of LOVD 2.0 started after the decision was made to rewrite all of LOVD from scratch
to be able to include a long list of upgrade suggestions that were hard to implement in LOVD 1.1.0.
Aimed at modularity and data redundancy, LOVD 2.0 was meant to be a more flexible and more powerful successor of the popular 1.1.0 version
and soon it received the interest of LOVD users eager to try out the all-new version.
\par
With more features being added and bugs fixed rapidly, LOVD 2.0 reached beta stage in April 2007,
after which more and more users started to upgrade their 1.1.0 databases to 2.0.
Finally, in October 2007 LOVD 2.0 reached the stable stage, after which LOVD 2.0 was continuously improved with monthly releases for two years,
after which the releases became less frequent.
LOVD 2.0 is decribed in the \href{http://www.ncbi.nlm.nih.gov/pubmed/21520333}{second LOVD paper}.
\\\par
By 2009 it had became clear that although LOVD 2.0 was a great step forward, there were still key improvements to be made.
Since the complexity of necessary changes had become to great to gradually upgrade LOVD 2.0 systems to include these options,
it was again decided to start from scratch writing LOVD 3.0.
This allowed us to redesign the complete data model in full freedom,
although it should still be possible for existing LOVD 2.0 databases to have all data transferred to LOVD 3.0.
\par
LOVD 3.0 adds even more flexibility, allowing users to focus exclusively on sequence variants,
whilst also allowing an exclusive focus on individuals and clinical data, and anything in between.
It will be possible for different submitters to work together cooperatively on the same data.
Searching through the data is improved extensively,
and webservices of many different sources are used to automatically retrieve gene and transcript information.
Also new in 3.0 is full Next Generation Sequencing (NGS) support,
with the ability to import VCF or SeattleSeq formats.
For VCF file imports, LOVD allows for automatic annotation of the variants.
\par
LOVD 3.0 reached beta stage in January 2012. Currently, the latest release is \LOVDversion.
Keep an eye on our \href{http://www.lovd.nl/3.0/news}{news page} for the latest information on LOVD 3.0 development.





%\chapter{Chapter...}



%\section{Section...}

%\subsection{Subsection...}

%\subsubsection{Subsubsection...}



%\begin{infobox}
%  \caption{\textbf{Caption}}
%  ...
%\end{infobox}

%\begin{notebox}
% FIXME; See http://en.wikibooks.org/wiki/LaTeX/Floats,_Figures_and_Captions for more options.
%...
%\end{notebox}





\end{document}
