\documentclass[a4paper,oneside,openany,12pt]{memoir}
\usepackage[T1]{fontenc} % To get different font encoding, thus allow \guillemotright.
% Undo bad "side effects" of T1 font encoding (ugly font && chapters in small caps).
% Note that it's now not shown in small caps simply because the selected font does not support it.
\usepackage{lmodern}
\usepackage{graphicx} % For images
\graphicspath{{../gfx/}}
%\usepackage[english]{babel} % For correct word hyphenating.
\usepackage{color}    % For colored links and boxes.
\usepackage[dvipsnames]{xcolor}
\definecolor{LOVDdark}{HTML}{224488}
\definecolor{LOVDlight}{HTML}{F0F3FF}
\usepackage{float}    % For custom floats (the info boxes).
\usepackage{wrapfig}  % For floating boxes meant for small notes.
% When loaded before float, doesn't work.
% For figures, maybe not do an frame but a box without border and with background color?
\usepackage[format=hang,font=footnotesize,labelfont=bf,skip=5pt]{caption} % To format captions.
\usepackage{hyperref} % For URLs.
\usepackage{tikz}
\usepackage{xr}

% Include all of this in a separate file!
\newcommand{\HRule}{\rule{\linewidth}{1mm}} % Doet height (zie style) hetzelfde?
\newcommand{\institute}[1]{\gdef\inst{#1}}  % Beamer supplies \institute. We want that, too.
\newcommand{\inst}{}                        % Beamer supplies \institute. We want that, too.
\newcommand{\funding}[1]{\gdef\fund{#1}}    % Provide funding line.
\newcommand{\fund}{}                        % Provide funding line.
\newcommand{\setcreativecommons}[1]{\gdef\creativecommons{#1}}	% Provide creative commons line.
\newcommand{\creativecommons}{}                       			 		% Provide creative commons line.
\newcommand{\setLOVDversion}[1]{\gdef\LOVDversion{#1}} % Provide the current LOVD version.
\newcommand{\LOVDversion}{}                            % Provide the current LOVD version.
\newcommand{\setpointercolor}[1]{\gdef\pointercolor{#1}}%
\newcommand{\pointercolor}{}  													%
\newcommand{\setpointerwidth}[1]{\gdef\pointerwidth{#1}}%
\newcommand{\pointerwidth}{}  													%
\newcommand{\setgrid}[1]{\gdef\drawgrid{#1}}						% Used for drawing a grid over a figure
\newcommand{\drawgrid}{} 																%
\newcommand{\setmanualversion}[1]{\gdef\manualversion{#1}} 	% Provide the current manual version.
\newcommand{\manualversion}{}                            		% Provide the current manual version.

\setlrmarginsandblock{2cm}{2cm}{*} % LEFT-RIGHT
\setulmarginsandblock{2cm}{2cm}{*} % TOP-BOTTOM
\checkandfixthelayout % Without this, nothing works. Took me ages before I found out.
\fixpdflayout % Not sure if we need this, but it was recommended someplace.



%%%%% PAGE HEADERS AND FOOTERS %%%%%
\makepagestyle{LOVD}

% Because we don't have odd or even pages, we only need to define odd pages.
\makeoddhead{LOVD}{\normalfont\leftmark}{}{\normalfont\rightmark}
\makeheadrule{LOVD}{\textwidth}{\normalrulethickness}
\makeoddfoot{LOVD}{}{\normalfont\thepage}{}
\makefootrule{LOVD}{\textwidth}{\normalrulethickness}{\footruleskip}

% Style "plain" is called from chapters. We want chapters to have a footer as well.
\makeoddfoot{plain}{}{\normalfont\thepage}{}
\makefootrule{plain}{\textwidth}{\normalrulethickness}{\footruleskip}

% Additional changes:
\makepsmarks{LOVD}{%
  \nouppercaseheads

  \createmark{chapter}{left}{shownumber}{}{.\space} % (\leftmark) number, followed by a . and a space.
  \createmark{section}{right}{nonumber}{}{.\space} % (\rightmark) no number, (useless: followed by a . and a space).
  % Change "shownumber" to "nonumber" if you don't want the chapter/section number displayed at the header.
}
% Activate your new pagestyle
\pagestyle{LOVD}



%%%%% TITLE PAGE FORMAT %%%%%
\setlength{\droptitle}{-3cm} % Moves the title (logo, title, authors etc) 3cm up.
\pretitle{
  \begin{center}
  	\includegraphics[width=16cm]{logo.jpg}
    \vskip 1.5cm
   	\includegraphics[width=3cm]{hgvs.png} \hspace{1.5cm}
		\includegraphics[width=3cm]{human_variome.png} \hspace{1.5cm}
    \includegraphics[width=3cm]{gen2phen.jpg}
    \vskip 3cm
    \HRule\par\HUGE\bfseries\sffamily} %% Need proper font!
\posttitle{\par\HRule\end{center}\vskip 1cm}
\preauthor{\flushright \vskip12em}
\postauthor{\par\inst\par\vskip 5mm}
\predate{\hfill  Last updated: }
\postdate{\par\clearpage \hfill \vskip50em \small \noindent \fund \creativecommons} 



%%%%% CHAPTER STYLE (CHAPTER HEADS) %%%%%
\makechapterstyle{LOVD}{%
  \setlength\beforechapskip{10pt} % A small distance just above the new Chapter title.
  \setlength\afterchapskip{20pt} % A small distance between the Chapter title and the text.
  \renewcommand{\chapterheadstart}{\vspace*{\beforechapskip}\hrule height 2pt \medskip} % Nice ruler above the Capter title.
  \renewcommand{\chapnamefont}{\normalfont\large\scshape} % CHAPTER
  \renewcommand{\chapnumfont}{\normalfont\huge\bfseries\scshape} % 1
  \renewcommand{\chaptitlefont}{\normalfont\huge\bfseries\scshape} % e.g. "Introduction"
  \renewcommand{\printchaptername}{} % Empty text instead of "Chapter".
                  \renewcommand{\chapternamenum}{ } % Weet niet wat dit anders doet.
                  \renewcommand{\printchapternum}{\chapnumfont \thechapter} % Weet niet wat dit anders doet.
  \renewcommand{\afterchapternum}{. } % Just a dot after the Chapter number, no new line.
  \renewcommand{\afterchaptertitle}{\par\nobreak\medskip\hrule\vskip\afterchapskip} % Nice ruler below the Capter title.
}
\chapterstyle{LOVD}



%%%%% LINK CONFIGURATION %%%%%
\definecolor{linkblue}{rgb}{0.1, 0, 1}
\hypersetup{
  colorlinks,
  citecolor=linkblue,
  filecolor=linkblue,
  linkcolor=linkblue,
  urlcolor=linkblue
}



%%%%% INFOTABLE AND WARNTABLE DEFINITIONS %%%%%
\newsavebox{\infobox}
\newlength{\infoboxlength}
\newlength{\infoboxinnerlength}
\setlength{\infoboxlength}{\textwidth}
\addtolength{\infoboxlength}{-2\fboxsep}
\addtolength{\infoboxlength}{-2\fboxrule}
\addtolength{\infoboxlength}{-1.7cm} % Manually configured value making sure the whole box doesn't exceed the line width.
\setlength{\infoboxinnerlength}{\infoboxlength}
\addtolength{\infoboxinnerlength}{-5pt} % Manually configured value making sure the text doesn't get too near the right border.

%%%%%% DEFINITIONS FOR \FBOX %%%%%%%
\setlength{\fboxsep}{2pt}%
\setlength{\fboxrule}{2pt}%

\newenvironment{infotable}
  {\begin{lrbox}{\infobox}%
    \begin{minipage}[t]{1.5cm}
      \centering
      \vspace{0pt}
      \includegraphics[width=1cm,height=1cm]{lovd_information.png}
    \end{minipage}
   \begin{minipage}[t]{\infoboxlength}\vspace{5pt}\begin{minipage}{\infoboxinnerlength}}
  {\vspace{6pt}\end{minipage}\end{minipage}\end{lrbox}%
   \begin{center}
   \fcolorbox{black}{LOVDlight}{\usebox{\infobox}}
   \end{center}}

\newenvironment{warntable}
  {\begin{lrbox}{\infobox}%
    \begin{minipage}[t]{1.5cm}
      \centering
      \vspace{0pt}
      \includegraphics[width=1cm,height=1cm]{lovd_warning.png}
    \end{minipage}
   \begin{minipage}[t]{\infoboxlength}\vspace{5pt}\begin{minipage}{\infoboxinnerlength}}
  {\vspace{6pt}\end{minipage}\end{minipage}\end{lrbox}%
   \begin{center}
   \fcolorbox{black}{LOVDlight}{\usebox{\infobox}}
   \end{center}}



%%%%% CONFIGURE LEFTBAR (FRAMED PACKAGE) %%%%%
% Taken and adapted from http://tex.stackexchange.com/questions/22526/regarding-the-leftbar-environment
% (thanks, xport & Martin Scharrer)
% I still don't like the space between the bar and the colorbox (can be removed by taking out the \hspace), but
% I want that space *inside* the colorbox.
\renewenvironment{leftbar}[1][\hsize]
{%
    \def\FrameCommand
    {%
        {\color{LOVDdark}\vrule width 3pt \hspace{5pt}}%
        \colorbox{LOVDlight}%
    }%
    \MakeFramed{\hsize#1\advance\hsize-\width\FrameRestore}%
}
{\endMakeFramed}



%%%%% CONFIGURE IMAGES %%%%%
\definecolor{shadecolor}{RGB}{240, 243, 255} %F0F3FF
\setgrid{\draw[help lines,xstep=.05,ystep=.05] (0,0) grid (1,1);
	\foreach \x in {0,1,...,9} { \node [anchor=north] at (\x/10,0) {0.\x}; }
	\foreach \y in {0,1,...,9} { \node [anchor=east] at (0,\y/10) {0.\y}; }}
%\newlength{\imagewidth}
%\setlength{\imagewidth}{\textwidth}
%\addtolength{\imagewidth}{-2\FrameSep}
%\addtolength{\imagewidth}{-2\FrameRule}

%%%%% SETTINGS FOR THE TITLE PAGE %%%%%
\institute{Leiden University Medical Center}
\funding{LOVD has received funding from the European Community's Seventh Framework Programme\\
  (FP7/2007-2013) under grant agreement no 200754 - the GEN2PHEN project.}
\setcreativecommons{
	\begin{flushleft}

	\includegraphics[width=3cm]{cc88x31.png} 
	\vskip1em
	\noindent This work is licensed under the Creative Commons
	Attribution-ShareAlike 4.0 International License. 
	To view a copy of this license, visit http://creativecommons.org/licenses/by-sa/4.0/
 	or send a letter to: Creative Commons, PO Box 1866, Mountain View, CA 94042, USA.
\end{flushleft}}


%%%%% Cross-referencing between different files %%%%%
\externaldocument[create_gene_]{Create_a_gene_variant_database}
\externaldocument[curate_gene_]{Curate_a_gene_variant_database}

%%%%%%%%%%%%%%%%%%%%%%%%%%%%%%%%%%%%%%%%%% NEW MAXIMUM LINE LENGTH (120 char) %%%%%%%%%%%%%%%%%%%%%%%%%%%%%%%%%%%%%%%%%%
